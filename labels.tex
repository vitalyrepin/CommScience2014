\documentclass[a4paper]{article}
\usepackage[utf8]{inputenc}
\usepackage[english,russian]{babel}
\usepackage{a4wide}
\setlength{\topmargin}{0pt}
\setlength{\headheight}{0pt}
\addtolength{\textheight}{4cm}
\voffset=-2cm
\pagestyle{empty}

\begin{document}
\Large

\newcommand{\hist}{
\begin{center}
\hrule
\medskip

\textbf{\textsc{Introduction to Communication Science}}
\smallskip

{\large\textrm{Rutger de Graaf (University of Amsterdam)}}

\smallskip

\textsl{Лето 2014\,г. Недели 1 -- 4}
\end{center}}

\hist

\large

\centerline{\parbox{14cm}{\begin{description}
\itemsep-.2cm
\item[Week 1] \textsl{Introduction}
\begin{itemize}
\item[1.1] Introduction (2:46)
\item[1.2] What is communication? (3:33)
\item[1.3] Concepts (3:41)
\item[1.4] Theories (4:15)
\item[1.5] Transmission (2:57)
\item[1.6] Reception, signs, signification (4:23)
\item[1.7] Cultural approach (4:03)
\item[1.8] Three approaches (2:27)
\end{itemize}
\item[Week 2] \textsl{A short history of communication science}
\begin{itemize}
\item[2.1] A short history of Communication Science (2:22)
\item[2.2] Greek and Roman Rhetorica (3:19)
\item[2.3] Two Schools of Classical Communication Science (3:30)
\item[2.4] Rhetorical Theory (5:03)
\item[2.5] The Dark Ages of Communication (4:28)
\item[2.6] A Renaissance of our field (3:29)
\item[2.7] The Printing Press as an Agent of Change (3:09)
\item[2.8] Towards a Modern Communication Science (7:03)
\end{itemize}
\item[Week 3] \textsl{The linear effect-oriented approach}
\begin{itemize}
\item[3.1] Introduction (3:04)
\item[3.2] The Power of Propaganda and the All-powerfull Media Paradigm (3:31)
\item[3.3] Needles, Bullets and Martians (3:46)
\item[3.4] Powerful Media put to the Test (5:12)
\item[3.5] Minimal effects (4:08)
\item[3.6] Powerful Media Rediscovered (4:35)
\item[3.7] A Revolution in the Media Landscape (4:40)
\item[3.8] Negotiated Effects (3:48)
\end{itemize}
\item[Week 4] \textsl{The reception and signification perspective}
\begin{itemize}
\item[4.1] An introduction to the Reception and Signification Perspective (3:51)
\item[4.2] Message Construction (4:12)
\item[4.3] Active Audiences (4:27)
\item[4.4] Selective Processing (3:20)
\item[4.5] Cognitive Shortcuts (4:09)
\item[4.6] Central and Peripheral Route (2:31)
\item[4.7] Getting Through the Filter (4:49)
\item[4.8] Encoding, Decoding and the Construction of Meaning (4:39)
\end{itemize}
\end{description}}}
\hrule
\newpage

\renewcommand{\hist}{
\begin{center}
\hrule
\medskip

\textbf{\textsc{Introduction to Communication Science}}
\smallskip

{\large\textrm{Rutger de Graaf (University of Amsterdam)}}


\smallskip

\textsl{Лето 2014\,г. Недели 5 -- 6}
\end{center}}

\hist

\centerline{\parbox{14cm}{\begin{description}
\itemsep-.2cm
\item[Week 5] \textsl{The cultural approach}
\begin{itemize}
\item[5.1] Introduction to the Cultural Approach (3:48)
\item[5.2] Producing and Maintaining Culture (4:07)
\item[5.3] Making Sense of the World (4:46)
\item[5.4] Popular Culture: Reflection or Illusion? (4:52)
\item[5.5] Cultural Studies: Birmingham and Toronto (5:42)
\item[5.6] Cultural Groups (4:33)
\item[5.7] How To Fit In? (5:05)
\item[5.8] Conclusion: Cultural Conformity and Relativism (4:21)
\end{itemize}
\item[Week 6] \textsl{Student questions and debate}
\begin{itemize}
\item[6.1] Introduction. Question \&\ Answers (2:10)
\item[6.2] Metaphors (3:45)
\item[6.3] The Role of Media in Society (4:24)
\item[6.4] More Metaphors on the Role of Media in Society (3:09)
\item[6.5] Primary and Secondary Research (5:02)
\item[6.6] What's New About New Media? (5:17)
\item[6.7] Globalization: Village vs. Tribes (5:34)
\end{itemize}
\end{description}}}

\hrule

\end{document}
